% ------------------ PACKAGES ------------------ 
% Packages add extra commands and features to your LaTeX document. 

% Various packages for maths:
\usepackage{amsmath,amsthm,amsfonts,amssymb,amscd, derivative, calc, dsfont, mathrsfs}

% Various packages for tables:
\usepackage{array, tabularx, booktabs, multicol}

% Various packages for text and lists: 
\usepackage{ragged2e, setspace, empheq, enumitem}

% Various packages for images:
\usepackage{float, caption, subcaption}

% The graphicx package provides graphics support for adding pictures.
\usepackage{graphicx}

% Longtable allows you to write tables that continue to the next page.
\usepackage{longtable}

% The geometry packages defines the page layout (page dimensions, margins, etc)
\usepackage[a4 paper, margin=1in, headsep=0.25in]{geometry}
% Control indents 
\setlength{\parindent}{0.3in}
% Control gap between paragraphs
\setlength{\parskip}{10pt}


% Defines the Font of the document, e.g. Latin Modern font (Check Fonts here: https://tug.org/FontCatalogue/)
\usepackage{lmodern}

% Font encoding
\usepackage[T1]{fontenc}

% This package allows the user to specify the input encoding e.g. accents for French, Spanish
\usepackage[utf8]{inputenc}

% This package allows you to add empty pages
\usepackage{emptypage}

% Provides control over the typography of the Table of Contents, List of Figures and List of Tables
\usepackage{tocloft}

% The setspace package controls the line spacing properties.
\usepackage{setspace}

% Allows the customization of Latex's title styles
\usepackage{titlesec}

% Allows the customization of Latex's table of contents title styles
\usepackage{titletoc}

% The package provides functions that offer alternative ways of implementing some LATEX kernel commands
\usepackage{etoolbox}

% Provides extensive facilities for constructing and controlling headers and footers
\usepackage{fancyhdr} 
\setlength{\headheight}{16pt}


% Typographical extensions, namely character protrusion, font expansion, adjustment 
%of interword spacing and additional kerning
\usepackage{microtype}

% Manages hyperlinks 
\usepackage[colorlinks=true,linkcolor=black,urlcolor=blue]{hyperref}

% Generates PDF bookmarks
\usepackage{bookmark}

% Add color to Tables
\usepackage[table,xcdraw]{xcolor}

% Use these two packages together -- they define symbols
%  for e.g. units that you can use in both text and math mode.
\usepackage{gensymb}
\usepackage{textcomp}

% Bibliography package
%\usepackage[backend=biber,style=nature]{biblatex}
% \addbibresource{references.bib} % Add the .bib file that contains the references



% This package provides an easy way to input latin sample text (for the template only)
\usepackage{blindtext}